
%% bare_jrnl.tex
%% V1.4a
%% 2014/09/17
%% by Michael Shell
%% see http://www.michaelshell.org/
%% for current contact information.
%%
%% This is a skeleton file demonstrating the use of IEEEtran.cls
%% (requires IEEEtran.cls version 1.8a or later) with an IEEE
%% journal paper.
%%
%% Support sites:
%% http://www.michaelshell.org/tex/ieeetran/
%% http://www.ctan.org/tex-archive/macros/latex/contrib/IEEEtran/
%% and
%% http://www.ieee.org/

%%*************************************************************************
%% Legal Notice:
%% This code is offered as-is without any warranty either expressed or
%% implied; without even the implied warranty of MERCHANTABILITY or
%% FITNESS FOR A PARTICULAR PURPOSE! 
%% User assumes all risk.
%% In no event shall IEEE or any contributor to this code be liable for
%% any damages or losses, including, but not limited to, incidental,
%% consequential, or any other damages, resulting from the use or misuse
%% of any information contained here.
%%
%% All comments are the opinions of their respective authors and are not
%% necessarily endorsed by the IEEE.
%%
%% This work is distributed under the LaTeX Project Public License (LPPL)
%% ( http://www.latex-project.org/ ) version 1.3, and may be freely used,
%% distributed and modified. A copy of the LPPL, version 1.3, is included
%% in the base LaTeX documentation of all distributions of LaTeX released
%% 2003/12/01 or later.
%% Retain all contribution notices and credits.
%% ** Modified files should be clearly indicated as such, including  **
%% ** renaming them and changing author support contact information. **
%%
%% File list of work: IEEEtran.cls, IEEEtran_HOWTO.pdf, bare_adv.tex,
%%                    bare_conf.tex, bare_jrnl.tex, bare_conf_compsoc.tex,
%%                    bare_jrnl_compsoc.tex, bare_jrnl_transmag.tex
%%*************************************************************************


% *** Authors should verify (and, if needed, correct) their LaTeX system  ***
% *** with the testflow diagnostic prior to trusting their LaTeX platform ***
% *** with production work. IEEE's font choices and paper sizes can       ***
% *** trigger bugs that do not appear when using other class files.       ***                          ***
% The testflow support page is at:
% http://www.michaelshell.org/tex/testflow/



\documentclass[journal]{IEEEtran}
%
% If IEEEtran.cls has not been installed into the LaTeX system files,
% manually specify the path to it like:
% \documentclass[journal]{../sty/IEEEtran}





% Some very useful LaTeX packages include:
% (uncomment the ones you want to load)


% *** MISC UTILITY PACKAGES ***
%
\usepackage{ifpdf}
% Heiko Oberdiek's ifpdf.sty is very useful if you need conditional
% compilation based on whether the output is pdf or dvi.
% usage:
% \ifpdf
%   % pdf code
% \else
%   % dvi code
% \fi
% The latest version of ifpdf.sty can be obtained from:
% http://www.ctan.org/tex-archive/macros/latex/contrib/oberdiek/
% Also, note that IEEEtran.cls V1.7 and later provides a builtin
% \ifCLASSINFOpdf conditional that works the same way.
% When switching from latex to pdflatex and vice-versa, the compiler may
% have to be run twice to clear warning/error messages.






% *** CITATION PACKAGES ***
%
\usepackage{cite}
% cite.sty was written by Donald Arseneau
% V1.6 and later of IEEEtran pre-defines the format of the cite.sty package
% \cite{} output to follow that of IEEE. Loading the cite package will
% result in citation numbers being automatically sorted and properly
% "compressed/ranged". e.g., [1], [9], [2], [7], [5], [6] without using
% cite.sty will become [1], [2], [5]--[7], [9] using cite.sty. cite.sty's
% \cite will automatically add leading space, if needed. Use cite.sty's
% noadjust option (cite.sty V3.8 and later) if you want to turn this off
% such as if a citation ever needs to be enclosed in parenthesis.
% cite.sty is already installed on most LaTeX systems. Be sure and use
% version 5.0 (2009-03-20) and later if using hyperref.sty.
% The latest version can be obtained at:
% http://www.ctan.org/tex-archive/macros/latex/contrib/cite/
% The documentation is contained in the cite.sty file itself.






% *** GRAPHICS RELATED PACKAGES ***
%
\ifCLASSINFOpdf
  \usepackage[pdftex]{graphicx}
  % declare the path(s) where your graphic files are
  % \graphicspath{{../pdf/}{../jpeg/}}
  % and their extensions so you won't have to specify these with
  % every instance of \includegraphics
  % \DeclareGraphicsExtensions{.pdf,.jpeg,.png}
\else
  % or other class option (dvipsone, dvipdf, if not using dvips). graphicx
  % will default to the driver specified in the system graphics.cfg if no
  % driver is specified.
  \usepackage[dvips]{graphicx}
  % declare the path(s) where your graphic files are
  % \graphicspath{{../eps/}}
  % and their extensions so you won't have to specify these with
  % every instance of \includegraphics
  % \DeclareGraphicsExtensions{.eps}
\fi
% graphicx was written by David Carlisle and Sebastian Rahtz. It is
% required if you want graphics, photos, etc. graphicx.sty is already
% installed on most LaTeX systems. The latest version and documentation
% can be obtained at: 
% http://www.ctan.org/tex-archive/macros/latex/required/graphics/
% Another good source of documentation is "Using Imported Graphics in
% LaTeX2e" by Keith Reckdahl which can be found at:
% http://www.ctan.org/tex-archive/info/epslatex/
%
% latex, and pdflatex in dvi mode, support graphics in encapsulated
% postscript (.eps) format. pdflatex in pdf mode supports graphics
% in .pdf, .jpeg, .png and .mps (metapost) formats. Users should ensure
% that all non-photo figures use a vector format (.eps, .pdf, .mps) and
% not a bitmapped formats (.jpeg, .png). IEEE frowns on bitmapped formats
% which can result in "jaggedy"/blurry rendering of lines and letters as
% well as large increases in file sizes.
%
% You can find documentation about the pdfTeX application at:
% http://www.tug.org/applications/pdftex





% *** MATH PACKAGES ***
%
\usepackage[cmex10]{amsmath}
% A popular package from the American Mathematical Society that provides
% many useful and powerful commands for dealing with mathematics. If using
% it, be sure to load this package with the cmex10 option to ensure that
% only type 1 fonts will utilized at all point sizes. Without this option,
% it is possible that some math symbols, particularly those within
% footnotes, will be rendered in bitmap form which will result in a
% document that can not be IEEE Xplore compliant!
%
% Also, note that the amsmath package sets \interdisplaylinepenalty to 10000
% thus preventing page breaks from occurring within multiline equations. Use:
%\interdisplaylinepenalty=2500
% after loading amsmath to restore such page breaks as IEEEtran.cls normally
% does. amsmath.sty is already installed on most LaTeX systems. The latest
% version and documentation can be obtained at:
% http://www.ctan.org/tex-archive/macros/latex/required/amslatex/math/





% *** SPECIALIZED LIST PACKAGES ***
%
\usepackage{algorithmic}
% algorithmic.sty was written by Peter Williams and Rogerio Brito.
% This package provides an algorithmic environment fo describing algorithms.
% You can use the algorithmic environment in-text or within a figure
% environment to provide for a floating algorithm. Do NOT use the algorithm
% floating environment provided by algorithm.sty (by the same authors) or
% algorithm2e.sty (by Christophe Fiorio) as IEEE does not use dedicated
% algorithm float types and packages that provide these will not provide
% correct IEEE style captions. The latest version and documentation of
% algorithmic.sty can be obtained at:
% http://www.ctan.org/tex-archive/macros/latex/contrib/algorithms/
% There is also a support site at:
% http://algorithms.berlios.de/index.html
% Also of interest may be the (relatively newer and more customizable)
% algorithmicx.sty package by Szasz Janos:
% http://www.ctan.org/tex-archive/macros/latex/contrib/algorithmicx/




% *** ALIGNMENT PACKAGES ***
%
\usepackage{array}
% Frank Mittelbach's and David Carlisle's array.sty patches and improves
% the standard LaTeX2e array and tabular environments to provide better
% appearance and additional user controls. As the default LaTeX2e table
% generation code is lacking to the point of almost being broken with
% respect to the quality of the end results, all users are strongly
% advised to use an enhanced (at the very least that provided by array.sty)
% set of table tools. array.sty is already installed on most systems. The
% latest version and documentation can be obtained at:
% http://www.ctan.org/tex-archive/macros/latex/required/tools/


% IEEEtran contains the IEEEeqnarray family of commands that can be used to
% generate multiline equations as well as matrices, tables, etc., of high
% quality.




% *** SUBFIGURE PACKAGES ***
\ifCLASSOPTIONcompsoc
  \usepackage[caption=false,font=normalsize,labelfont=sf,textfont=sf]{subfig}
\else
  \usepackage[caption=false,font=footnotesize]{subfig}
\fi
% subfig.sty, written by Steven Douglas Cochran, is the modern replacement
% for subfigure.sty, the latter of which is no longer maintained and is
% incompatible with some LaTeX packages including fixltx2e. However,
% subfig.sty requires and automatically loads Axel Sommerfeldt's caption.sty
% which will override IEEEtran.cls' handling of captions and this will result
% in non-IEEE style figure/table captions. To prevent this problem, be sure
% and invoke subfig.sty's "caption=false" package option (available since
% subfig.sty version 1.3, 2005/06/28) as this is will preserve IEEEtran.cls
% handling of captions.
% Note that the Computer Society format requires a larger sans serif font
% than the serif footnote size font used in traditional IEEE formatting
% and thus the need to invoke different subfig.sty package options depending
% on whether compsoc mode has been enabled.
%
% The latest version and documentation of subfig.sty can be obtained at:
% http://www.ctan.org/tex-archive/macros/latex/contrib/subfig/




% *** FLOAT PACKAGES ***
%
%\usepackage{fixltx2e}
% fixltx2e, the successor to the earlier fix2col.sty, was written by
% Frank Mittelbach and David Carlisle. This package corrects a few problems
% in the LaTeX2e kernel, the most notable of which is that in current
% LaTeX2e releases, the ordering of single and double column floats is not
% guaranteed to be preserved. Thus, an unpatched LaTeX2e can allow a
% single column figure to be placed prior to an earlier double column
% figure. The latest version and documentation can be found at:
% http://www.ctan.org/tex-archive/macros/latex/base/


%\usepackage{stfloats}
% stfloats.sty was written by Sigitas Tolusis. This package gives LaTeX2e
% the ability to do double column floats at the bottom of the page as well
% as the top. (e.g., "\begin{figure*}[!b]" is not normally possible in
% LaTeX2e). It also provides a command:
%\fnbelowfloat
% to enable the placement of footnotes below bottom floats (the standard
% LaTeX2e kernel puts them above bottom floats). This is an invasive package
% which rewrites many portions of the LaTeX2e float routines. It may not work
% with other packages that modify the LaTeX2e float routines. The latest
% version and documentation can be obtained at:
% http://www.ctan.org/tex-archive/macros/latex/contrib/sttools/
% Do not use the stfloats baselinefloat ability as IEEE does not allow
% \baselineskip to stretch. Authors submitting work to the IEEE should note
% that IEEE rarely uses double column equations and that authors should try
% to avoid such use. Do not be tempted to use the cuted.sty or midfloat.sty
% packages (also by Sigitas Tolusis) as IEEE does not format its papers in
% such ways.
% Do not attempt to use stfloats with fixltx2e as they are incompatible.
% Instead, use Morten Hogholm'a dblfloatfix which combines the features
% of both fixltx2e and stfloats:
%
% \usepackage{dblfloatfix}
% The latest version can be found at:
% http://www.ctan.org/tex-archive/macros/latex/contrib/dblfloatfix/




%\ifCLASSOPTIONcaptionsoff
%  \usepackage[nomarkers]{endfloat}
% \let\MYoriglatexcaption\caption
% \renewcommand{\caption}[2][\relax]{\MYoriglatexcaption[#2]{#2}}
%\fi
% endfloat.sty was written by James Darrell McCauley, Jeff Goldberg and 
% Axel Sommerfeldt. This package may be useful when used in conjunction with 
% IEEEtran.cls'  captionsoff option. Some IEEE journals/societies require that
% submissions have lists of figures/tables at the end of the paper and that
% figures/tables without any captions are placed on a page by themselves at
% the end of the document. If needed, the draftcls IEEEtran class option or
% \CLASSINPUTbaselinestretch interface can be used to increase the line
% spacing as well. Be sure and use the nomarkers option of endfloat to
% prevent endfloat from "marking" where the figures would have been placed
% in the text. The two hack lines of code above are a slight modification of
% that suggested by in the endfloat docs (section 8.4.1) to ensure that
% the full captions always appear in the list of figures/tables - even if
% the user used the short optional argument of \caption[]{}.
% IEEE papers do not typically make use of \caption[]'s optional argument,
% so this should not be an issue. A similar trick can be used to disable
% captions of packages such as subfig.sty that lack options to turn off
% the subcaptions:
% For subfig.sty:
% \let\MYorigsubfloat\subfloat
% \renewcommand{\subfloat}[2][\relax]{\MYorigsubfloat[]{#2}}
% However, the above trick will not work if both optional arguments of
% the \subfloat command are used. Furthermore, there needs to be a
% description of each subfigure *somewhere* and endfloat does not add
% subfigure captions to its list of figures. Thus, the best approach is to
% avoid the use of subfigure captions (many IEEE journals avoid them anyway)
% and instead reference/explain all the subfigures within the main caption.
% The latest version of endfloat.sty and its documentation can obtained at:
% http://www.ctan.org/tex-archive/macros/latex/contrib/endfloat/
%
% The IEEEtran \ifCLASSOPTIONcaptionsoff conditional can also be used
% later in the document, say, to conditionally put the References on a 
% page by themselves.




% *** PDF, URL AND HYPERLINK PACKAGES ***
%
\usepackage{url}
% url.sty was written by Donald Arseneau. It provides better support for
% handling and breaking URLs. url.sty is already installed on most LaTeX
% systems. The latest version and documentation can be obtained at:
% http://www.ctan.org/tex-archive/macros/latex/contrib/url/
% Basically, \url{my_url_here}.




% *** Do not adjust lengths that control margins, column widths, etc. ***
% *** Do not use packages that alter fonts (such as pslatex).         ***
% There should be no need to do such things with IEEEtran.cls V1.6 and later.
% (Unless specifically asked to do so by the journal or conference you plan
% to submit to, of course. )


% correct bad hyphenation here
\hyphenation{op-tical net-works semi-conduc-tor}

% My packages
\usepackage{comment}
% \usepackage[ruled]{algorithm2e}

% My mods:
\newcommand{\papertitle}{Constructing the Pareto Front using Limited Information: A Case of the Opposition based Solution Generation Scheme}
\graphicspath{{/media/khaled/data/research/oemo/onsga2r/experiments/results/}}

\begin{document}
%
% paper title
% Titles are generally capitalized except for words such as a, an, and, as,
% at, but, by, for, in, nor, of, on, or, the, to and up, which are usually
% not capitalized unless they are the first or last word of the title.
% Linebreaks \\ can be used within to get better formatting as desired.
% Do not put math or special symbols in the title.
% \title{Bare Demo of IEEEtran.cls for Journals}
\title{\papertitle}
%
%
% author names and IEEE memberships
% note positions of commas and nonbreaking spaces ( ~ ) LaTeX will not break
% a structure at a ~ so this keeps an author's name from being broken across
% two lines.
% use \thanks{} to gain access to the first footnote area
% a separate \thanks must be used for each paragraph as LaTeX2e's \thanks
% was not built to handle multiple paragraphs
%

\author{Author~1,~\IEEEmembership{Member,~IEEE,}
        Author~2,~\IEEEmembership{Fellow,~OSA,}
        and~Author~3,~\IEEEmembership{Life~Fellow,~IEEE}% <-this % stops a space
% \thanks{M. Shell is with the Department of Electrical and Computer Engineering, Georgia Institute of Technology, Atlanta, GA, 30332 USA e-mail: (see http://www.michaelshell.org/contact.html).}% <-this % stops a space
\thanks{Author-1 and Authtor-2 are with Anonymous University.}% <-this % stops a space
% \thanks{Manuscript received April 19, 2005; revised September 17, 2014.}%
}

% note the % following the last \IEEEmembership and also \thanks - 
% these prevent an unwanted space from occurring between the last author name
% and the end of the author line. i.e., if you had this:
% 
% \author{....lastname \thanks{...} \thanks{...} }
%                     ^------------^------------^----Do not want these spaces!
%
% a space would be appended to the last name and could cause every name on that
% line to be shifted left slightly. This is one of those "LaTeX things". For
% instance, "\textbf{A} \textbf{B}" will typeset as "A B" not "AB". To get
% "AB" then you have to do: "\textbf{A}\textbf{B}"
% \thanks is no different in this regard, so shield the last } of each \thanks
% that ends a line with a % and do not let a space in before the next \thanks.
% Spaces after \IEEEmembership other than the last one are OK (and needed) as
% you are supposed to have spaces between the names. For what it is worth,
% this is a minor point as most people would not even notice if the said evil
% space somehow managed to creep in.



% The paper headers
%\markboth{Journal of \LaTeX\ Class Files,~Vol.~13, No.~9, September~2014}%
%{Shell \MakeLowercase{\textit{et al.}}: Bare Demo of IEEEtran.cls for Journals}
% The only time the second header will appear is for the odd numbered pages
% after the title page when using the twoside option.
% 
% *** Note that you probably will NOT want to include the author's ***
% *** name in the headers of peer review papers.                   ***
% You can use \ifCLASSOPTIONpeerreview for conditional compilation here if
% you desire.




% If you want to put a publisher's ID mark on the page you can do it like
% this:
%\IEEEpubid{0000--0000/00\$00.00~\copyright~2014 IEEE}
% Remember, if you use this you must call \IEEEpubidadjcol in the second
% column for its text to clear the IEEEpubid mark.



% use for special paper notices
%\IEEEspecialpapernotice{(Invited Paper)}




% make the title area
\maketitle

% As a general rule, do not put math, special symbols or citations
% in the abstract or keywords.
\begin{abstract}
In this paper we investigate a curious example of opposition based solution generation applied to an evolutionary multi-objective optimization (EMO) algorithm, namely on NSGA-II \(\ldots\)
\end{abstract}

% Note that keywords are not normally used for peerreview papers.
\begin{IEEEkeywords}
IEEEtran, journal, \LaTeX, paper, template.
\end{IEEEkeywords}






% For peer review papers, you can put extra information on the cover
% page as needed:
% \ifCLASSOPTIONpeerreview
% \begin{center} \bfseries EDICS Category: 3-BBND \end{center}
% \fi
%
% For peerreview papers, this IEEEtran command inserts a page break and
% creates the second title. It will be ignored for other modes.
\IEEEpeerreviewmaketitle



\section{Introduction}
% The very first letter is a 2 line initial drop letter followed
% by the rest of the first word in caps.
% 
% form to use if the first word consists of a single letter:
% \IEEEPARstart{A}{demo} file is ....
% 
% form to use if you need the single drop letter followed by
% normal text (unknown if ever used by IEEE):
% \IEEEPARstart{A}{}demo file is ....
% 
% Some journals put the first two words in caps:
% \IEEEPARstart{T}{his demo} file is ....
% 
% Here we have the typical use of a "T" for an initial drop letter
% and "HIS" in caps to complete the first word.
% \IEEEPARstart{T}{his} demo file is intended to serve as a ``starter file''
% for IEEE journal papers produced under \LaTeX\ using
% IEEEtran.cls version 1.8a and later.

\IEEEPARstart{T}{alk} about some introductory stuffs and do some literature reviews to set the scenario, also discuss basic things like ``what is MOP'' etc. 

% You must have at least 2 lines in the paragraph with the drop letter
% (should never be an issue)
% I wish you the best of success.

% \hfill mds
 
% \hfill September 17, 2014

\section{An Alternative Interpretation of Opposition}
\label{sec:alternative-interpretation}
As we have already seen in the previous section, mostly the idea of \textit{opposition} is employed as the incorporation of new solutions with a certain kind of \textit{opposite traits} into the existing population \cite{?}. Such \textit{traits} could be interpreted in terms different perspectives. For example, an \textit{opposite} solution could be -- i) the one with an opposite representation (with respect to the current best point) \cite{?}, ii) the ones with the opposite values from the other spectrum of the variable bounds (i.e. in the case of real valued optimization) \cite{?}. However, upon injecting the opposite solutions could cause a re-route from the continuing search trajectory and thus could be a misleading step -- in a sense that the opposite solutions could only be useful if the search space follows a desired pattern \cite{?}. 

For example, if the task is to solve the N-queen problem \cite{?}, then the opposite representation of the current best can result into another valid global optima. One can easily verify this fact by computing the reverse assignment of queens from a existing optimal solution and it will eventually take us to another global optima. This observation also assumes that the search space also needs to be multi-modal, following from the fact that the reverse representation of the current best solution needs to be an optimum for another peak of the search space. Therefore, most of the standard opposition based algorithms inject the opposite solutions during the optimization start-up \cite{?}; or maintain a constant (generally low) ratio of opposite points \cite{?} into the current population. Therefore, the standard opposite injection scheme could only be effective given the two main assumptions on the underlying search space are valid -- multi-modality and the solution symmetry. For this reason, such scheme is quite unwieldy to incorporate into a large scale numerical optimization problems, e.g. multi-objective optimization problems (MOPs) \cite{?}.

Moreover, we have also found some examples of opposition based algorithm for Q-learning/TD-learning like scenarios \cite{?}. Similar argument can be made, as the reinforcement-learning algorithms are inherently greedy \cite{?} (as they rely on the Bellman's optimality principle). And the opposite actions during the learning phase introduces a noise; so that the search can branch out to alternative choices \cite{?}. In that sense, we can say that the injection of opposite solutions can be considered as a different form of \textit{variation operator} in population based stochastic search. 

However, for the case of MOPs, existing opposition based solution generation scheme may not be effective just because of the sheer complexity of the search space. Moreover, we hardly have any room to make any assumption about the multi-modality and/or symmetry of the solution representation (i.e. such assumption could be made about N-queen problem). Therefore, in this paper, we have revised the notion of opposition in terms of the algorithm's behaviour. For example, most Evolutionary Multi-objective Optimization (EMO) algorithms aim to maximize two principal properties -- i) the convergence and ii) the diversity, as the quality of a MOP solution depends on these two factors \cite{?}. Therefore, this change of notion will now allows us to re-consider the opposite point generation/injection in a different perspective --
\begin{itemize}
	\item Convergence: A solution \textit{far} from the true Pareto-front is \textit{opposite} to any solution that is \textit{closer} to the true Pareto-front.
	\item Diversity: An \textit{isolated} solution on the true Pareto-front is \textit{opposite} to a \textit{crowded} solution. 
\end{itemize}

By taking the above two principles into account, we will deterministically generate opposite solutions during the search. Obviously, the deterministic point generation scheme will only consider the \textit{opposite trait} that is \textit{good}. In the next section we will see, how the existing EMO algorithms shows the limitations maintaining this two \textit{opposite traits} during the search (i.e. solution generation) process.  

\section{Limitations with the Canonical MOP Algorithms: The Search Trajectory Bias}
\label{sec:limitation-canonical}
Most of the standard EMO algorithms (e.g. NSGA-II, SPEA-II etc.), are elitist by design \cite{?}\cite{?}. They are also ``opportunistic'' in a sense that the population always try to converge to a particular portion of the Pareto-front (PF) which seems to be easier to solve at that point. They also shows a preference over a certain objective function which needs less exploration than the other. We can see such biasness in the search when we try to solve the ZDT4 problem using NSGA-II. In such case, the first objective is easier to optimize than the second one, the readers can verify this fact from the figure \ref{fig:zdt4-unbalanced-snapshot}. Therefore, the search trajectory deliberately accumulates more points over the first objective to optimize one particular portion of the Pareto-front. Moreover, while putting more solutions to the vicinity of one particular objective axis, the search trajectory looses the uniformity by forming a crowded streak of points along that axis; on the other hand we can see that there is almost no solution on the other spectrum of the objective space. This kind of non-symmetric search behaviour is, we think, causes a hindrance to the optimization algorithm. Therefore, it would be helpful if we could selectively inject points during the search where the solution distribution is more sparse. In addition, we also think that this biased nature of the search trajectory could degrade with the addition of more objective functions. Moreover, this could also lead to a stagnation on the local optima, given that the search space has a lot many of them.
%
% Figure
\begin{figure}[tp]
\centering
\includegraphics[width=0.33\textwidth]{acmlarge-mouse}
\caption{The effect of the \textit{search trajectory bias} could be seen when we try to solve ZDT4 problem with NSGA-II. Here we can see a long streak of crowded solution near the objective \(f_1\), where the distribution of solutions near the objective \(f_2\) is extremely sparse.}
\label{fig:zdt4-unbalanced-snapshot}
\end{figure}

Given this specific scenario, now the main problem is to devise a way to deterministically generate points where the distribution of the solution is sparse. Here we assume that the lesser the number of solutions in a vicinity of objective \(f_i\), the harder it is to solve. This would be easy if we know the exact mapping of the design variable to objective values -- however such mapping is always unavailable and above all it is very expensive to infer \cite{?}. Another way could be to mutate the points where the solution is more sparse, but we think as the original algorithm already goes through such step, it is not going to be very effective \footnote{However, in the section \ref{?} we will demonstrate this fact that just mutating the sparse solutions does not help much.}.

Therefore, in this paper we are going to demonstrate a very effective approach to address this issue, we will show how we can maintain a balanced distribution of solutions that is parallel to the true PF. The proposed approach is also extremely effective in converging fast to the true PF as well.   

\section{The Deterministic Opposite Point Generation Scheme}
\label{sec:generation-scheme}
As we have discussed in the section \ref{sec:alternative-interpretation}, we will utilize the so-called notion of \textit{Opposition} to deterministically generate points on the strategically useful place on the search space. In principle, we do not assume any exact mapping over the design variables to objective values, and we apply a linear approximation to achieve our goal effectively \footnote{As a matter of fact, we will also see that such opposite points are generally 30\% useful in most cases and they are more effective during the initial generations -- which is also a very interesting finding.}. 

Our basic approach is to infer the true PF before starting the actual optimization run. To do this, we depend on barely \(k\) number (\(k = \text{number of objectives}\)) of extreme (or near extreme) points on the true PF since we can safely assume that the population will eventually reach to the vicinity of those extreme points in the end. Moreover, the extreme points will be used as a pivot to arbitrate the \textit{opposite} traits over the existing solutions. Also note that we are not going to deterministically define which portion of the search space is less easy (or hard) and so forth -- we will try to devise a technique that will automatically address and solve such issues on the way.

Another reason to fixate over the \(k\) extreme points is that we also wanted to keep the algorithm simple so that it can only utilize the ``minimal information'' of the true PF. We also think it's valid to assume that any PF could be bounded by at least \(k\)-extreme points for any \(k\)-objective problem. Although, if we could supply other intermediary points on the true PF, we will be able to see a better performance gain with the existing model, however supply of 1 extra true PF solution comes with an added cost of extra function evaluations. As the extreme (or near extreme) points are the pivot to define the notion of \textit{opposite} in our case, we will start the next section by discussing how to find them efficiently --

\subsection{Finding the Extreme Points}
Extreme points on the Pareto-front could be found using global search as well \cite{}, however our goal was to save the extra computational cost as much as possible. Therefore, we resort to classical single-objective optimization methods to solve this problem. Our choice of such algorithms were limited to, namely, the Interior Point Method\footnote{We have used the \texttt{fmincon()} routine in MATLAB (v. R2014a) for this purpose.} and Mesh Adaptive Direct Search\footnote{We have used the \texttt{patternsearch()} routine in MATLAB (v. R2014a) for this purpose.}. Depending on the difficulty of the problems, appropriate routine parameters were empirically found out and they are summarized as follows: 
%
\begin{itemize}
	\item If the number of variables $> 10$, use \texttt{fmincon()} with default options.
	\item Otherwise use \texttt{patternsearch()} with default options, however if the number of objective $> 2$, then use these settings:
		\begin{itemize}
			\item \texttt{InitialMeshSize:} population size
			\item \texttt{TolX:} $1e^{-7}$, \texttt{TolBind:} $1e^{-6}$
			\item \texttt{Search Method:} \texttt{@MADSPositiveBasis2N} starts searching with $2N$ random directions, where $N =$ number of variables.
			\item \texttt{CompletePoll:} \texttt{on} and \texttt{CompleteSearch:} \texttt{on}
		\end{itemize}
\end{itemize}
%
In general, for problems with less than \(10\) variables were solved using MADS and the larger problems were solved using IPM. The readers should be aware that the algorithm parameters that we have found is not universal setting, they are subjected to empirical investigations and problem domain-knowledge. 
% \medskip
% \begin{algorithm}[H]
% 	\SetAlgoNoLine
% 	$k \leftarrow$ no. of objectives for a particular problem\\
% 	$T \leftarrow (\text{population size} \times (\frac{1}{4} \times \text{maximum generation})) \times \frac{1}{k}$\\
% 	$E^\ast \leftarrow \{\}$, an empty solution set\\
% 	\For{all $i$ from $1$ to $k$}
% 	{
% 		$f_i \leftarrow$ $i$-th objective function\\
% 		$x \leftarrow $ random initial vector\\
% 		\While{maximum of $\frac{T}{2}$ function evaluation reached}
% 		{
% 			\If{$|x| > 10$}
% 			{
% 				$x \leftarrow$ solve $f_i$ with \texttt{fmincon()}
% 			}
% 			\Else
% 			{
% 				$x \leftarrow$ solve $f_i$ with \texttt{patternsearch()}
% 			}
% 		}
% 		$x \leftarrow$ solution found from the above optimization loop\\
% 		$f_{\text{aasf}} \leftarrow $ construct AASF function from $f_i$\\
% 		\While{maximum of $\frac{T}{2}$ function evaluation reached}
% 		{
% 			\If{$|x| > 10$}
% 			{
% 				$x \leftarrow$ solve $f_{\text{aasf}}$ with \texttt{fmincon()}
% 			}
% 			\Else
% 			{
% 				$x \leftarrow$ solve $f_{\text{aasf}}$ with \texttt{patternsearch()}
% 			}
% 		}
% 		$E^\ast \leftarrow E^\ast \cup \{x\}$\\
% 		return $E^\ast$ \\
% 	}
% 	\caption{Find Extreme Points ()}
%     	\label{algo:extreme-point}
% \end{algorithm}
% \medskip

The actual extreme point location algorithm was conducted in two steps -- given a particular objective function \(f_i\), first we try to solve it directly using either IPM or MADS (depending on the problem type); then after some \(T_1\) iterations, we construct the Augmented Achievement Scalarizing Function (AASF) function from \(f_i\) and solve it again for \(T_2\) iterations. To limit the function evaluations, we kept \(T = T_1 + T_2\) to a constant value. For all problems, we have fixed this maximum iteration count to the \(\frac{1}{4}\)-th of the total generation specified. A basic listing for this routine is presented in algorithm \ref{algo:extreme-point}. The set of the extreme points \(E = \{E^\ast\}\) generated from this algorithm may not contain all unique solution, and also they might not be the true extreme always. However, our algorithm can still converge to the true PF extremes.

\subsection{The Opposite Solution Generation Algorithm}
\label{sec:main-algo}
Once the extreme points are discovered, now we utilize them to generate the so-called \textit{opposite} points during the main evolutionary runs. On each generation, we randomly select 30\% solutions from the current population and deterministically change them to generate opposite solutions to place them in strategically viable place. And to conduct this variation, we will utilize the points in the set \(E\) as pivot points. We call these points as ``pivot'' since we will selectively try to generate points around these pivots. However, before doing this, we will \textit{refine} our pivot points \(E = \{E^\ast\}\) in a certain way. 

The \textit{refinement} starts by finding the current population extreme points \(E_c\) and merging them with the set \(E\) such that \(E = \{E_c \cup E^\ast\}\). Next we apply the non-dominated sort on \(E\) to find the Pareto-front within this set. We apply this sorting to keep the true extreme points if ones are found in the later generations; and also if \(E^\ast\) are happen to be the weakly dominated points. After this step, we select the points from \(E\) that are on the best front and with \(\infty\) crowding distances\footnote{Here, by ``crowding distance'', we mean the inter-solution distances as computed in NSGA-II.}. Lets denote these selected points as \(E'\). Now at this point, only two kinds of situations are possible:
%
\begin{itemize}
	\item The set \(E'\) contains only the solutions \(E^\ast\) if we are in the initial generations, or
	\item The \(E'\) will contain the solutions \(E_c\) if we are in the later phase of the generations, where \(E_c\) will be the true PF extreme. 
\end{itemize}
%
% \medskip
% \begin{algorithm}[H]
% 	\SetAlgoNoLine
% 	$E \leftarrow$ true PF extremes $E^\ast$ found from \texttt{FIND-EXTREME-POINTS()}\\
% 	$E_c \leftarrow$ find the extreme points from the current best front in population $P$\\
% 	$E \leftarrow \{E^\ast \cup E_c\}$\\
% 	$E' \leftarrow$ rank points in $E$ and select the extremes from the best front\\
% 	\For{all points $p_i$ in $E - E'$}
% 	{
% 		\If{$p_i$ weakly dominates any solution $p_j \in E'$}
% 		{
% 			replace $p_j$ by $p_i$
% 		}
% 	}
% 	update $E^\ast$, $E^\ast \leftarrow E'$\\
% 	$G \leftarrow$ find $k$ intermediary gap points from the current best front in population $P$\\
% 	$E' \leftarrow \{E' \cup G\}$\\
% 	return $E'$\\
% 	\caption{Generate Pivot Points ($P$, $E^\ast$)}
%     	\label{algo:main-algo}
% \end{algorithm}
% \medskip

However, during the intermediary generations, it could also happen that we might include some solutions into \(E\) that weakly dominate some points already in \(E\), this inclusion will reduce the expected spread of the pivot points -- that may diminish the effect of maintaining the diversity. Therefore, if there exist a point in \(E - E'\) that is on the best front and also weakly dominated by any point in \(E'\), then replace the weakly dominating point from \(E'\) with the one from the set \(E - E'\). The readers might have already noticed that \(|E'| \le |E|\). \vfill \eject

Now, at this point, we can ensure that the set \(E'\) contains either true PF extremes or points near them. Now if we can generate new points near \(E'\), they will induce both better convergence and diversity. In section \ref{sec:limitation-canonical}, we have discussed a scenario where we can see how the biasness in the search trajectory is introduced. However, the difference in the relative difficulty of the objective functions may not be the only reason for such bias, the imbalance in the solution distribution could happen for other reasons as well. For example, a disconnected Pareto-front, a local optimal front or a specific portion of the Pareto-front being more difficult to solve than the rest. In such cases, we can see a \textit{gap} forming over the Pareto-front during the search, we can see such a convergence pattern in many problems. To address these \textit{gaps}, we also find the solutions with \(k\)-highest (\(k = \) no. of objectives) crowding distance from the best front that are not \(\infty\), and denote them as \(G\). Clearly, these \(G\) solutions are those that reside on the edge of the broken Pareto-front. Now we add the \(G\) to the set \(E'\), thus we make \(E'\) as the final ``pivot'' solutions to generate the opposite points. This should also be noted that \(|E'| > k\).
%
% \medskip
% \begin{algorithm}[H]
% 	\SetAlgoNoLine
% 	$t = 1$, $N \leftarrow$ population size $|P_t|$\\
% 	$E^\ast \leftarrow$ call \texttt{FIND-EXTREME-POINTS()} to get the true PF extremes\\
% 	\While{$t \le$ some maximum generation}
% 	{
% 		$P'_t \leftarrow$ randomly select 30\% solution from $P_t$\\
% 		$E'_t \leftarrow$ call \texttt{GENERATE-PIVOT-POINTS($P_t$, $E^\ast$)} to make the pivot set\\
% 		$O_t \leftarrow \emptyset$\\
% 		\For{each solution $x_i \in P'_t$}
% 		{
% 			$S \leftarrow$ pick $k$ random solutions from $E'_t$\\
% 			$v \leftarrow$ $v \in S$ such that it is the furthest from $x_i$\\
% 			$x_c \leftarrow$ generate opposite point from $v$ and $x_i$\\
% 			$O_t \leftarrow \{O_t \cup x_c\}$
% 		}
% 		$P_t \leftarrow \{P_t \cup  E^\ast\}$, $R_t \leftarrow P_t \cup Q_t$\\
% 		$\mathcal{F} \leftarrow$ \texttt{fast-non-dominated-sort(}$R_t$\texttt{)}\\
% 		$P_{t+1} \leftarrow \emptyset$, $i \leftarrow 1$\\
% 		\While{$|P_{t+1}| + |\mathcal{F}_i| \le N$}
% 		{
% 			\texttt{crowding-distance-assignment(}$\mathcal{F}_i$\texttt{)}\\
% 			$P_{t+1} \leftarrow P_t + \mathcal{F}_i$\\
% 			$i \leftarrow i + 1$
% 		}
% 		sort $\mathcal{F}_i$ in descending order using $\prec_n$\\
% 		$P_{t+1} \leftarrow$ the first $N - |P_{t+1}|$ solutions from $\mathcal{F}_i$\\
% 		$Q_{t+1} \leftarrow$ \texttt{crossover-mutate(}$P_{t+1}$\texttt{)}\\
% 		\For{each solution $x_i \in O_t$}
% 		{
% 			randomly insert $x_i$ into $Q_{t+1}$
% 		}
% 		$t \leftarrow t + 1$
% 	}
% 	\caption{NSGA-II with Opposition}
%     	\label{algo:onsga2}
% \end{algorithm}
% \medskip

As we have mentioned at the beginning that we randomly select 30\% of the current population for opposite point generation. We go through each of them and every time we randomly pick \(k\) number of random points from \(E'\) and pick the pivot point that is the furthest from it, and find the opposite vector using a linear scaling. Instead of any sophistication, we do this scaling in a more straight-forward way -- given a pivot vector \(\mathbf{v}\) and a parent vector \(\mathbf{x_p}\), we generate an \textit{opposite} child \(\mathbf{x_c}\) as \(\mathbf{x_c} = U(\frac{3}{4}||\mathbf{x_c} - \mathbf{x_p}||,\frac{5}{4}||\mathbf{x_c} - \mathbf{x_p}||)\mathbf{x_p}\), where \(U(d,u)\) is a uniform random number within the range \([d,u]\). The overall procedure is presented in algorithm \ref{algo:main-algo}. When we apply this algorithm to NSGA-II, we follow the obvious way, the generated opposite population will be inserted into the child population \(Q_t\) in the NSGA-II run, the algorithm \ref{algo:onsga2} shows how to call this procedure in NSGA-II like algorithm. Moreover, this algorithm is ``pluggable'' in a sense that we can integrate it to any other elitist EMO algorithms.

\section{Experiments with the Multi-objective Problem Sets}
Intro to this section.

\section{NSGA-II Equipped with Extreme Points}
Discuss if NSGA-II is given two extreme points, how it behaves (less robust)

\subsection{ZDT Problems}
Discuss ZDT problem set results.

\subsection{DTLZ Problems}
Discuss DTLZ problem set results.

\subsection{Constrained and Rotated Problems}
Discuss constrained and rotated problem (OSY, DTLZ8) results.

\section{NSGA-II Compensated for the Extra Function Evaluations}
Discuss what if we run the NSGA-II with the compensated function evaluations.  

\subsection{ZDT Problems}
Discuss ZDT problem set results.

\subsection{DTLZ Problems}
Discuss DTLZ problem set results.

\subsection{Constrained and Rotated Problems}
Discuss constrained and rotated problem (OSY, DTLZ8) results.

\section{Conclusions and Future Works}
Discuss.

\begin{comment}
\subsection{Subsection Heading Here}
Subsection text here.

% needed in second column of first page if using \IEEEpubid
%\IEEEpubidadjcol

\subsubsection{Subsubsection Heading Here}
Subsubsection text here.


% An example of a floating figure using the graphicx package.
% Note that \label must occur AFTER (or within) \caption.
% For figures, \caption should occur after the \includegraphics.
% Note that IEEEtran v1.7 and later has special internal code that
% is designed to preserve the operation of \label within \caption
% even when the captionsoff option is in effect. However, because
% of issues like this, it may be the safest practice to put all your
% \label just after \caption rather than within \caption{}.
%
% Reminder: the "draftcls" or "draftclsnofoot", not "draft", class
% option should be used if it is desired that the figures are to be
% displayed while in draft mode.
%
%\begin{figure}[!t]
%\centering
%\includegraphics[width=2.5in]{myfigure}
% where an .eps filename suffix will be assumed under latex, 
% and a .pdf suffix will be assumed for pdflatex; or what has been declared
% via \DeclareGraphicsExtensions.
%\caption{Simulation results for the network.}
%\label{fig_sim}
%\end{figure}

% Note that IEEE typically puts floats only at the top, even when this
% results in a large percentage of a column being occupied by floats.


% An example of a double column floating figure using two subfigures.
% (The subfig.sty package must be loaded for this to work.)
% The subfigure \label commands are set within each subfloat command,
% and the \label for the overall figure must come after \caption.
% \hfil is used as a separator to get equal spacing.
% Watch out that the combined width of all the subfigures on a 
% line do not exceed the text width or a line break will occur.
%
%\begin{figure*}[!t]
%\centering
%\subfloat[Case I]{\includegraphics[width=2.5in]{box}%
%\label{fig_first_case}}
%\hfil
%\subfloat[Case II]{\includegraphics[width=2.5in]{box}%
%\label{fig_second_case}}
%\caption{Simulation results for the network.}
%\label{fig_sim}
%\end{figure*}
%
% Note that often IEEE papers with subfigures do not employ subfigure
% captions (using the optional argument to \subfloat[]), but instead will
% reference/describe all of them (a), (b), etc., within the main caption.
% Be aware that for subfig.sty to generate the (a), (b), etc., subfigure
% labels, the optional argument to \subfloat must be present. If a
% subcaption is not desired, just leave its contents blank,
% e.g., \subfloat[].


% An example of a floating table. Note that, for IEEE style tables, the
% \caption command should come BEFORE the table and, given that table
% captions serve much like titles, are usually capitalized except for words
% such as a, an, and, as, at, but, by, for, in, nor, of, on, or, the, to
% and up, which are usually not capitalized unless they are the first or
% last word of the caption. Table text will default to \footnotesize as
% IEEE normally uses this smaller font for tables.
% The \label must come after \caption as always.
%
%\begin{table}[!t]
%% increase table row spacing, adjust to taste
%\renewcommand{\arraystretch}{1.3}
% if using array.sty, it might be a good idea to tweak the value of
% \extrarowheight as needed to properly center the text within the cells
%\caption{An Example of a Table}
%\label{table_example}
%\centering
%% Some packages, such as MDW tools, offer better commands for making tables
%% than the plain LaTeX2e tabular which is used here.
%\begin{tabular}{|c||c|}
%\hline
%One & Two\\
%\hline
%Three & Four\\
%\hline
%\end{tabular}
%\end{table}


% Note that the IEEE does not put floats in the very first column
% - or typically anywhere on the first page for that matter. Also,
% in-text middle ("here") positioning is typically not used, but it
% is allowed and encouraged for Computer Society conferences (but
% not Computer Society journals). Most IEEE journals/conferences use
% top floats exclusively. 
% Note that, LaTeX2e, unlike IEEE journals/conferences, places
% footnotes above bottom floats. This can be corrected via the
% \fnbelowfloat command of the stfloats package.

\end{comment}


\section{Conclusion}
The conclusion goes here.





% if have a single appendix:
%\appendix[Proof of the Zonklar Equations]
% or
%\appendix  % for no appendix heading
% do not use \section anymore after \appendix, only \section*
% is possibly needed

% use appendices with more than one appendix
% then use \section to start each appendix
% you must declare a \section before using any
% \subsection or using \label (\appendices by itself
% starts a section numbered zero.)
%


\appendices
\section{Proof of the First Zonklar Equation}
Appendix one text goes here.

% you can choose not to have a title for an appendix
% if you want by leaving the argument blank
\section{}
Appendix two text goes here.


% use section* for acknowledgment
\section*{Acknowledgment}


The authors would like to thank...


% Can use something like this to put references on a page
% by themselves when using endfloat and the captionsoff option.
\ifCLASSOPTIONcaptionsoff
  \newpage
\fi



% trigger a \newpage just before the given reference
% number - used to balance the columns on the last page
% adjust value as needed - may need to be readjusted if
% the document is modified later
%\IEEEtriggeratref{8}
% The "triggered" command can be changed if desired:
%\IEEEtriggercmd{\enlargethispage{-5in}}

% references section

% can use a bibliography generated by BibTeX as a .bbl file
% BibTeX documentation can be easily obtained at:
% http://www.ctan.org/tex-archive/biblio/bibtex/contrib/doc/
% The IEEEtran BibTeX style support page is at:
% http://www.michaelshell.org/tex/ieeetran/bibtex/
%\bibliographystyle{IEEEtran}
% argument is your BibTeX string definitions and bibliography database(s)
%\bibliography{IEEEabrv,../bib/paper}
%
% <OR> manually copy in the resultant .bbl file
% set second argument of \begin to the number of references
% (used to reserve space for the reference number labels box)
\begin{thebibliography}{1}

\bibitem{IEEEhowto:kopka}
H.~Kopka and P.~W. Daly, \emph{A Guide to \LaTeX}, 3rd~ed.\hskip 1em plus
  0.5em minus 0.4em\relax Harlow, England: Addison-Wesley, 1999.

\end{thebibliography}

% biography section
% 
% If you have an EPS/PDF photo (graphicx package needed) extra braces are
% needed around the contents of the optional argument to biography to prevent
% the LaTeX parser from getting confused when it sees the complicated
% \includegraphics command within an optional argument. (You could create
% your own custom macro containing the \includegraphics command to make things
% simpler here.)
%\begin{IEEEbiography}[{\includegraphics[width=1in,height=1.25in,clip,keepaspectratio]{mshell}}]{Michael Shell}
% or if you just want to reserve a space for a photo:

% \begin{IEEEbiography}{Michael Shell}
% Biography text here.
% \end{IEEEbiography}

% if you will not have a photo at all:
% \begin{IEEEbiographynophoto}{John Doe}
% Biography text here.
% \end{IEEEbiographynophoto}

% insert where needed to balance the two columns on the last page with
% biographies
%\newpage

% \begin{IEEEbiographynophoto}{Jane Doe}
% Biography text here.
% \end{IEEEbiographynophoto}

% You can push biographies down or up by placing
% a \vfill before or after them. The appropriate
% use of \vfill depends on what kind of text is
% on the last page and whether or not the columns
% are being equalized.

%\vfill

% Can be used to pull up biographies so that the bottom of the last one
% is flush with the other column.
%\enlargethispage{-5in}



% that's all folks
\end{document}


